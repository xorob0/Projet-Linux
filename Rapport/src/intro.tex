\section{Introduction}
\subsection{Choix de la distribution}
Pour le choix de la distribution nous avons commencé pas mettre en place certains critères de recherches:
\begin{itemize}
	\item La gratuité
	\item La stabilité
	\item La légèreté
\end{itemize}
Nos recherches nous montrent plusieurs choix possibles:
\begin{itemize}
	\item RHEL : Payante
	\item Arch Linux : Pas la plus stable car elle fonctionne en rolling release
	\item Ubuntu Server et Debian : Ne possèdant pas de version core ce qui les rend plus lourdes
	\item CentOS : Répond au mieux à tout nos critères
\end{itemize}
Nous allons donc faire notre serveur sous CentOS car celui ci répond à tous nos critères dans sa version Core.

\subsection{Mode d'installation}
Pour l'installation nous avons opté pour l'écriture de scripts pour chacune des fonctionnalités de notre serveur. Cela nous permet de nous rappeller de nos procédures et de toujours les comprendre dans quelques années.
Il nous suffit maintenant de copier nos scripts sur une machine réelle ou virtuelle pour pouvoir immédiatement commencer à configurer Linux.
Bien évidement nos scripts ne gèrent pas beaucoup d'erreurs et ne vérifient pas ce que l'utilisateur a entré, ils ne sont donc pas vraiment prêt pour tous les usages mais sont très utiles comme notes.

\subsection{Organisation du groupe}
Pour nous organiser nous avons utilisé les outils de GitHub. Nous avons donc commencé par créer un repo (privé pour le moment mais nous le passerons surement en publique après les examens) nous permettant de travailler ensemble sur notre code. Nous avons aussi utilisé la ToDo list de GitHub pour nous organiser dans notre développement.

\subsection{Machine physique}
Nous avons eu l'occation d'utiliser une machine physique pour mettre en pratique nos scripts. Celle-ci est bien évidement une machine de récupération, elle est composée de Intel Pentium 4, de 1Go de DDR2 et de deux disques dur en RAID 1.
