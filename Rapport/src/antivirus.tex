\section{Anti-virus}
\lstinputlisting[language=Bash]{../scripts/antivirus/antivirus.sh}
Ce script va permettre d’installer l’antivirus sur le serveur pour amélioré sa sécurité
On vérifie d’abord que l’utilisateur exécute bien le script en tant que root, ce qui est nécessaire pour configurer les fichiers de ClamAV.
On installe ensuite EPEL (Extra Package for Entreprise Linux) qui est un repo fournissant des package additionnels pour les distributions de type RedHat/CentOs.

Ensuite, au cas où on ne possède pas de fichier de configuration, on copie le template clamd.conf. On peut créer le service freshclam  et le configurer de sortie qu’il vérifie 4 fois par jour la précense de mise à jour. On le démarre et on l’active au démarrage ensuite.

On renomme également le fichier /usr/lib/systemd/system/clamd@.service en /usr/lib/systemd/system/clamd.service autrement, par défaut le service ne fonctionne pas. Puis, il faut indiquer le bon chemin du clamd.service dans le fichier /usr/lib/systemd/system/clamd@scan.service et on peut configurer le fichier clamd.service
Les dernières lignes vont pemettre d’activer et de démarrer automatiquement les services clamd et clamd@scan au démarrage

\lstinputlisting[language=Bash]{../scripts/antivirus/analyseVirus.sh}

Ce script va permettre d’exécuter une analyse virale d’un dossier ou fichier choisis par l’utilisateur qui lance le script.
On met d’abord à jour la base de données de l’antivirus et puis on demande à l’utilisateur de spécifier la cible. Si elle existe, on procède à l’analyse.
