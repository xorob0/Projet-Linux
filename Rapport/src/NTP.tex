\section{Serveur NTP}
Le protocole NTP (\textit{\textbf{N}etwork \textbf{T}ime \textbf{P}rotocol}) va permettre de synchroniser les horloges des ordinateurs connectés au même réseau local que celle du serveur de temps.
Celui-ci devra synchroniser sa propre horloge en contactant un serveur de temps de référence à distance donc par internet.
Cette synchronisation des heures permettra entre autres de ne pas perturber certaines applications utilisant l’horloge du système mais aussi pour donner plus de cohérence en cas de comparaison des messages de « logs » de plusieurs ordinateurs sur le réseau.

\lstinputlisting[language=Bash]{../scripts/ntp/NTP.sh}
\lstinputlisting[language=Bash]{../scripts/ntp/ntp.conf}

On déclare d’abord le fichier de dérive « driftfile ». Il va permettre de corriger les dérives de l’horloge système en l’absence de connexion réseau au serveur de référence.
On déclare ensuite le répertoire et le fichier pour stocker les « logs » du service ntpd. 
On permet la synchronisation avec notre source de temps mais on interdit à la source de modifier ou d’interroger le service sur ce système.
On autorise les accès sur l’interface de bouclage
Ensuite, on se synchronise avec les serveurs NTP belge de référence. Pour résoudre les problèmes de charges on entre plusieurs adresses. Il s’agit d’un groupement de serveur et la redistribution se fait à l’aide du Round Robin DNS (association de plusieurs adresses IP à un FQDN)
Enfin, on précise au serveur de se synchroniser sur l’ « Undisciplined Local Clock » et on indique un faux « pilote » destiné à la sauvegarde de l’heure dans le cas où aucune source externe d’heure synchronisée n’est disponible.
